%second chapter of your thesis
\chapter{Coach}
Dit jaar was het de eerste keer dat alle teams een persoonlijke coach werden toegekend. Tijdens de projectweek moesten we onze zwakke punten opgeven. Naar gelang deze zwakke punten hebben we een persoonlijke begeleider gekregen, in ons geval Kevin. Wij hadden vooraf vooral schrik voor het PCB-design en het kunnen verklaren van het al dan niet weglaten van bepaalde componenten en het waarom daarvan. Kevin stond altijd klaar om op onze vragen te antwoorden en als het nodig was ons te helpen zoeken naar het probleem. Dit hadden we vooral nodig voor het installeren van de Ardumoto voor de motoraansturing. Er zaten enkele foutjes in. Kevin ontzag het niet om enkele uren te helpen zoeken naar de fout in de PCB, die we na lang zoeken dan eindelijk gevonden hebben. Kortom, voor ons was een persoonlijke coach een zeer goede ervaring. We konden altijd terecht bij de hoofdbegeleiders Guus en Stijn, maar het was altijd handig om bij meer specifieke vragen te kunnen vragen aan Kevin.