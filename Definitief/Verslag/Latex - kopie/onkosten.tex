%second chapter of your thesis
\chapter{Onkosten}
In dit hoofdstuk hebben we een kleine tabel gemaakt van onze aangekochte componenten/modules. Alle basiscomponenten die we nodig hadden voor de ArduMoto werd sowieso al voorzien en moesten we dus niet opnemen in deze onkosten tabel. Hier moeten we enkel onze eigen specifieke aankopen vermelden. Zoals ook al vermeld in hoofdstuk~\ref{chap:Opdrachtbeschrijving} konden we ook in principe onbeperkt 3D-stukken printen. Dit hebben we ook een paar keer gebruikt maar dit moest ook niet in de onkosten gezet worden. We hebben dus slechts drie verschillende zaken aangekocht, namelijk een bluetooth-module, een RFID-module en 10 line-following sensors. Dit alles hebben we voorgesteld in tabel~\ref{table:Onkosten}.\\
\begin {table}[H]
\caption {Tabel met aankopen en hun prijs.} \label{table:Onkosten}
\begin{center}
	\begin{tabular}{ | l | l | }
	%\label{table:Onkosten}
	\hline
	Beschrijving aankoop & Prijs (in euro) \\ \hline
	\hline
	HCO5 (Bluetooth module) & 5.37 \\ \hline
	MRFC522 (RFID Reader) & 4.15 \\ \hline
	10x Line-following Sensors & 2.16 \\ \hline
	\end{tabular}
\end{center}
\end{table}
