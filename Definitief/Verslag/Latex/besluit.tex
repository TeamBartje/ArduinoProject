%second chapter of your thesis
\chapter{Besluit}
We kunnen terugkijken op een zeer leerrijke en toch wel zeer plezante bachelorproef. De twee aspecten van Elektronica zaten erin verwerkt, nl. hardware maar ook een belangrijk deel software. We hebben ook enkele testjes uit moeten voeren om bijvoorbeeld te beslissen hoe we de sensors moesten plaatsen, ver/dicht van het wagentje, schuin of recht, … Dan moesten we ook zelf beslissen welke modules we gebruikten, wat we wel en wat niet zelf moesten maken. We hebben toch wel de meeste foutjes en problemen tegengekomen die iedereen in een leerproces moet meemaken. Bij ieder nieuw PCB-design zitten er wel altijd enkele foutjes, dit was in ons geval natuurlijk niet anders. Sommige problemen konden we zeer snel en gemakkelijk oplossen, andere problemen hebben bloed, zweet en tranen gekost om te vinden en op te lossen. Bij het software onderdeel was het vooral de uitdaging om de PID (in ons geval PD) afregeling te doen, vooral omdat we dit nog in geen enkel vak in theorie hadden gezien. We kunnen wel met een positieve blik terugkijken op onze vormgeving en afregeling van ons wagentje. We waren over alle vier de parcours die we voorgeschoteld kregen de snelste, we moeten er wel bij vertellen dat we 4x uit het parcours zijn geraakt. Twee keer meer dan de tweede groep die in het totaal 2x uit het parcours ontsnapte. Kortom uit deze bachelorproef hebben we volgende zaken geleerd: PID afregelen, Bluetooth connectie maken met RPI, RFID’s uitlezen, PCB ontwerpen, PCB debuggen, SMD solderen, 3D-printen. 
