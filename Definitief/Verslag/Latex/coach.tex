%second chapter of your thesis
\chapter{Coach}
Dit jaar was het de eerste keer dat aan elk team een persoonlijke coach werd toegekend bij wie we konden langsgaan indien we met vragen zaten over de uitwerking van het project. Tijdens de projectweek moesten we nadenken over onze zwakke punten en en onze sterktes en aan de hand daarvan hebben we een persoonlijke begeleider gekregen, in ons geval Kevin. Wij hadden vooraf vooral schrik voor het PCB-design en het kunnen verklaren van het al dan niet weglaten van bepaalde componenten. Kevin stond altijd klaar om op onze vragen te antwoorden en als het nodig was ons te helpen zoeken naar het probleem. Dit hadden we vooral nodig voor het installeren van de L298 voor de motoraansturing waar er nog enkele foutjes in zaten. Kevin ontzag het niet om enige tijd te helpen zoeken naar de fout in de PCB, die we uiteindelijk hebben kunnen vinden en oplossen. Kortom, voor ons was een persoonlijke coach een zeer goede ervaring. We konden altijd terecht bij de hoofdbegeleiders Guus en Stijn, maar het was altijd handig om bij meer specifieke vragen te kunnen vragen aan Kevin.